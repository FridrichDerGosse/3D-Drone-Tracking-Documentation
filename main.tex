%% Template basiert auf der Vorlage der Uni Graz für VWA: https://latex.tugraz.at/vorlagen/allgemein

%% Versionen:
%% V1: 9. Augugst 2021 (GreiA)


\input{template/main_settings}

%% ========================================================================
%% Document metadata: DIESE WERTE BITTE ANPASSEN, wie werden dann automatisch auf der
%% Titelseite angezeigt
%% ========================================================================

\newcommand{\mytitle}{3D Drone Tracking} 
\newcommand{\mysubtitle}{Image-Driven 3D Drone Tracking employing Multiple Stations for Agricultural Use}
\newcommand{\myinstitute}{Abteilung für Elektronik und Technische Informatik} 
\newcommand{\mysubmissionyear}{2025} %% Einreich - Jahr
\newcommand{\mysubmissionmonth}{March} %% Monat der Einreichung
\newcommand{\myauthor}{Prantl Niclas\\Krahbichler Lukas}  %% Autoren. Bitte mit \\ Trennen wenn mehrere
\newcommand{\mysupervisor}{
	MMag.{}^{a} \textnormal{ Egger Eva-Maria, MA}
	\\Dipl.-Ing. Götsch Leopold, PMM
	\\Mag. \ phil.\ Jank Andreas}  %%Betreuer. Bitte mit \\ Trennen wenn mehrere
\newcommand{\myprojectpartner}{None}  %% Partnerfirma

\newcommand{\mysubject}{SUBJECT}  %% also used for PDF metadata (hyperref)
\newcommand{\mykeywords}{KEYWORDS}  %% also used for PDF metadata (hyperref)


%% header settings
\usepackage{lastpage}

\ohead{\headmark }
\ihead*{\includegraphics[width=3cm]{figures/htl-logo}}

\ifoot{\thepage}  %Will man Anzahl Seiten: /\pageref{LastPage}
\ofoot{\myauthor}


%% ========================================================================
%%%% MISC command definitions
%% ========================================================================
\input{template/mycommands}

%% ========================================================================
%%%% Typographic settings
%% ========================================================================
\input{template/typographic_settings}
\input{template/listing_format}

%% ========================================================================
%%%% MISC usepackages
%% ========================================================================

%% ... it's OK to put here your own usepackage commands ...

\usepackage{listings}
\usepackage{xcolor}
\usepackage{wrapfig}
\usepackage[acronym]{glossaries}
\usepackage{graphicx}

%\usepackage[hyphens]{url}            % Improves URL line breaking
%\usepackage[hidelinks]{hyperref}     % Optional: disables colored boxes/links
%\usepackage{etoolbox}
%\appto\UrlBreaks{\do\_\do\/}         % Allow breaks at _ and /

%% ========================================================================
%%%% MISC self-defined commands and settings
%% ========================================================================

%% ... it's OK to put here your own newcommand/newenvironment-definitions ...

\lstdefinestyle{PythonStyle}{
	tabsize=4,
	language=Python,
	basicstyle=\ttfamily,
	keywordstyle=\color{blue}\bfseries,
	stringstyle=\color{red},
	commentstyle=\color{gray},
	numberstyle=\tiny\color{gray},
	numbers=left,
	stepnumber=1,
	showstringspaces=false,
	breaklines=true,
	frame=single
}

\lstdefinestyle{CppStyle}{
	tabsize=4,
	language=C++,
	basicstyle=\ttfamily,
	keywordstyle=\color{purple}\bfseries,
	stringstyle=\color{orange},
	commentstyle=\color{gray},
	numberstyle=\tiny\color{gray},
	numbers=left,
	stepnumber=1,
	showstringspaces=false,
	breaklines=true,
	frame=single
}



\hyphenation{ex-am-ple hy-phen-ate}  %% in order to use German umlauts
%% here (Ver-\"of-fent-li-chung), you have to check for
%% activated \usepackage[T1]{fontenc} in the preamble

%% override default language of babel: (be sure to know, what you're
%% doing here)
\selectlanguage{american}
%\selectlanguage{ngerman}

%% ========================================================================
%% bibtex für die Literaturverwaltung: Hier wird der Zitier-Stil festgelegt
%% ========================================================================
\usepackage[square,numbers]{natbib}
\bibliographystyle{agsm} 


\input{template/pdf_settings}  %% should be *last* definitions in preamble!


\makeglossaries

\newacronym{3d}{3D}{Three-Dimensional}
\newacronym{gnss}{GNSS}{Global navigation satellite system}
\newacronym{rtk}{RTK}{Real-Time Kinematic}
%\newacronym{ai}{AI}{Artificial Intelligence}
% WIFI, LiDAR, 

%% ========================================================================
%%%% begin{document}
%% ========================================================================

\let\origcleardoublepage\cleardoublepage
\let\cleardoublepage\clearpage

% Festlegen des grundlegenden Layouts (Ränder auch in cm angebbar)
% \usepackage[oneside, left=1in, right=1in, top=1in, bottom=1in]{geometry}

\begin{document} 
	
\frontmatter                    %% KOMA: roman page numbers and such; only available in scrbook

%% \input{colophon}                %% defines information about editor, LaTeX, font, ...

%% Choose your desired title page:
\input{\mytitlepage}            %% include title page

% Jetzt schalten wir wieder zurück auf twopage
%\newgeometry{twoside, left=1in, right=1in, top=1in, bottom=1in, bindingoffset=0.5in}

%\input{template/lock_flag} % Wenn kein Sperrvermerk gemacht werden soll, dann diesen Import einfach auskommentieren

%%\input{template/declaration_TU_Graz}  %% Statutory Declaration
% \input{thanks}                %% this is a suggestion: you have to create this file on demand
% \input{foreword}               %% this is a suggestion: you have to create this file on demand


%% include the abstract without chapter number but include it on table of contents:
\phantomsection
\addcontentsline{toc}{chapter}{Abstract}
\chapter*{Abstract / Kurzfassung}
\label{cha:abstract}

This diploma thesis explores the development of a ground-based 3D drone tracking system that does not rely on expensive hardware installed inside each drone. Three camera-equipped ground stations capture images of drones in flight, calculate their positions in three-dimensional space, and present the results in a user-friendly interface. The project’s main goal is to provide a more affordable alternative for smaller-scale applications—such as agricultural use—where traditional drone technology may be cost-prohibitive.

Central to this approach are synchronized calibration routines, a approximation algorithm for calculating positions, and locally managed data processing. Early experiments confirmed that relative angles and known station coordinates can, in principle, deliver valid tracking data. However, hardware issues—particularly with the chosen mini-computer—prevented the completion of a fully integrated prototype.

Despite these setbacks, the concept shows clear potential for drone tracking without requiring additional modules or external networks. Future work could address hardware reliability and weatherproof housing. Overall, this system lays the groundwork for a cost-effective alternative to conventional drone tracking methods and opens up new possibilities for broader adoption in budget-sensitive areas.

\vspace{1cm}

Diese Diplomarbeit beschäftigt sich mit der Entwicklung eines bodengestützten 3D-Drohnen-Ortungssystems, das keine kostspieligen Bauteile in der Drohne selbst benötigt. Drei Kamerastationen am Boden erfassen die Fluggeräte, berechnen deren räumliche Position und visualisieren das Ergebnis in einer benutzerfreundlichen Oberfläche. Hauptziel ist es, insbesondere in kleineren Anwendungsbereichen—wie etwa der Landwirtschaft—eine erschwingliche Alternative zu schaffen, bei der herkömmliche Drohnen-Technik oft zu teuer wäre.

Das Konzept basiert auf synchronisierten Kalibrierverfahren, einem Algorithmus zur Positionsbestimmung und lokaler Datenverarbeitung in den Bodenstationen. Erste Versuche zeigten, dass Winkelmessungen und bekannte relative Stationskoordinaten prinzipiell verlässliche Ortungsdaten liefern können. Allerdings führten Hardwareprobleme—insbesondere mit dem eingesetzten Mini-Computer—dazu, dass kein vollständig integrierter Prototyp realisiert werden konnte.

Trotz dieser Schwierigkeiten unterstreicht unsere Arbeit, dass eine Drohnenortung ohne zusätzliche Module oder externe Netzwerke grundsätzlich möglich ist. Zukünftige Arbeiten könnten sich auf zuverlässigere Hardware und ein wetterfestes Gehäuse konzentrieren. Insgesamt legt diese Arbeit den Grundstein für eine preisgünstige Alternative zu gängigen Ortungs-Systemen und eröffnet neuen Nutzergruppen den Zugang zur Drohnentechnologie.
              %% Abstract


\chapter*{Declaration of Independent Work}
\label{cha:affirmation}


\textbf{STATUTORY DECLARATION}
\vspace{1cm}

I hereby declare under oath that I have composed the present work independently and without external assistance, have not used any sources and aids other than those specified, and have identified as such the passages extracted literally and in content from the sources used. My work may be made publicly accessible if there is no confidentiality notice.

\vfill\vfill



Place, Date  \hspace{5cm}Author 1     
\vfill    
Place, Date  \hspace{5cm}Author 2        %%EIDESSTATTLICHE ERKLÄRUNG  

\tableofcontents                %% this produces the table of contents - you might have guessed :-)



%% if myaddlistoftodos is set to "true", the current list of open todos is added:
\ifthenelse{\boolean{myaddlistoftodos}}{
  \newpage\listoftodos          %% handy if you are using todonotes with \todo{}
}{}                             %% with todonotes-package option "disable" you can get rid of any todo in the output

\mainmatter                     %% KOMA: marks main part using arabic page numbers and such; only available in scrbook

\let\cleardoublepage\origcleardoublepage


%% HIER DIE EIGENEN KAPITEL EINFÜGEN
\chapter{Introduction}

In the introduction, it is explained why this topic was chosen. (Objectives and tasks of the overall project, technical and economic environment)

\section{Detailed Task Description}

\subsection{Student Name 1}

\subsection{Student Name 2}

\section{Documentation of the Work}

The project results are documented.

\begin{itemize}
	\item Basic concept
	\item Theoretical foundations
	\item Practical implementation
	\item Solution approach
	\item Alternative solution approach
	\item Results including interpretation
\end{itemize}

Further suggestions:

\begin{itemize}
	\item Manufacturing documents
	\item Test cases (measurement results...)
	\item User documentation
	\item Used technologies and development tools
\end{itemize}
 
 
\chapter{Market Analysis}

\subsection{Industry Overview and Market Potential}

The proposed 3D object tracking system addresses the needs of several rapidly growing industries with substantial market potential:

\begin{itemize}
	\item \textbf{Industrial Automation:} Precise tracking is crucial in manufacturing and logistics for process optimization and safety. 3D tracking integrated into industrial automation enhances accuracy in assembly lines, warehouse management, and quality control. As industries progress toward Industry 4.0, the need for sophisticated tracking solutions is growing.
	
	\begin{table}[H]
		\centering
		\label{table:industrial_automation}
		\begin{tabular}{lrrr}
			\hline\hline
			\textbf{Source} & \textbf{Year Range} & \textbf{CAGR} & \textbf{Market Value (USD)} \\
			\hline
			\cite{alliedmarketresearch} & 2021-2031 & 8.7\% & 196.4 B - 443.5 B \\
			\cite{grandviewresearch} & 2023-2030 & 10.5\% & 172.26 B - 377.25 B \\
			\cite{mordorintelligence} & 2024-2029 & 8.77\% & 203.05 B - 309.16 B \\
			\hline
		\end{tabular}
		\caption{Industrial Automation Market Projections}
	\end{table}
	
	\item \textbf{Augmented and Virtual Reality (AR/VR):} 3D tracking is fundamental for AR and VR applications, providing accurate spatial awareness and interaction. These technologies are used in gaming, education, and remote collaboration, enhancing user experiences through immersive environments.
	
	\begin{table}[H]
		\centering
		\label{table:ar_vr}
		\begin{tabular}{lrrr}
			\hline\hline
			\textbf{Source} & \textbf{Year Range} & \textbf{CAGR} & \textbf{Market Value (USD)} \\
			\hline
			\cite{statista_ar_vr} & 2024-2029 & 8.97\% & 40.4 B - 62 B \\
			\cite{precedenceresearch} & 2023-2032 & 22.9\% & 23.92 B - 187.28 B \\
			\cite{theinsightpartners_ar_vr} & 2023-2031 & 36.9\% & 52.4 B - 646.5 B \\
			\hline
		\end{tabular}
		\caption{AR/VR Market Projections}
	\end{table}
	
	\item \textbf{Security Systems:} 3D tracking enhances security by providing real-time location data, crucial for monitoring high-security environments. It improves situational awareness and response times by tracking people and objects.
	
	\begin{table}[H]
		\centering
		\label{table:security}
		\begin{tabular}{lrrr}
			\hline\hline
			\textbf{Source} & \textbf{Year Range} & \textbf{CAGR} & \textbf{Market Value (USD)} \\
			\hline
			\cite{databridgemarketresearch} & 2021-2028 & 11.5\% &  - 100.35 B \\
			\cite{statista_security} & 2022-2027 & - & 130.07 B - 234.72 B \\
			\cite{theinsightpartners_security} & 2023-2031 & 10.4\% & 51.78 B to 114.55 B \\
			\hline
		\end{tabular}
		\caption{Video Surveillance Market Projections}
	\end{table}
	
	\item \textbf{Autonomous Vehicles and Drones:} Accurate 3D tracking is vital for navigating autonomous vehicles and drones, especially in complex environments. It enhances obstacle detection, path planning, and decision-making, essential for safe operations.
	
	\begin{table}[H]
		\centering
		\label{table:drones}
		\begin{tabular}{lrrr}
			\hline\hline
			\textbf{Source} & \textbf{Year Range} & \textbf{CAGR} & \textbf{Market Value (USD)} \\
			\hline
			\cite{statista_drones} & 2024-2029 & 2.24\% & 4.3 B - ? \\
			\cite{mordorintelligence_drones} & 2024-2029 & 13.74\% & 17.31 B - 32.95 B\\
			\cite{expertmarketresearch} & 2024-2032 & 10.8\% & 27.7 B - 59.2 B\\
			\hline
		\end{tabular}
		\caption{Drone Market Projections}
	\end{table}
	
\end{itemize}

These industries, encompassing manufacturers, security providers, and technology companies, represent a vast potential customer base. The increasing demand for automation, safety, and immersive experiences underscores the significant market opportunity for a versatile 3D object tracking solution.

\subsection{Target Group Definition}

Our ideal customers are mid-size to large enterprises in the following sectors:

\begin{itemize}
	\item Industrial Automation
	\item Augmented and Virtual Reality (AR/VR)
	\item Security and Surveillance
	\item Autonomous Vehicles and Drones
\end{itemize}

\textbf{Key Characteristics}

\begin{itemize}
	\item \textbf{Demographics}: Technical decision-makers (35-55), advanced degrees, primarily male, often fitting the 'Performer' and 'Adaptive Pragmatic Center' Sinus Milieus \cite{sinus_institut_2024}.
	\item \textbf{Geographics}: Based in developed economies, primarily urban tech hubs (e.g., Vienna, Linz, ...).
	\item \textbf{Psychographics}: Innovation-driven, seeking efficiency and precision; value data-driven decisions.
	\item \textbf{Behavioral}: Conduct extensive research, prefer proven, scalable solutions with reliable vendor support.
	\item \textbf{Needs}: Require systems that enhance tracking precision, integrate seamlessly, and provide cost-effective solutions.
	\item \textbf{Technographics}: Technologically sophisticated, highly active online, engaged in industry-specific digital platforms.
\end{itemize}

We estimate a potential customer base of several hundred companies, primarily in tech-savvy, high-growth markets.

\subsection{Buyer Personas}

\textbf{Core Persona 1: Industrial Automation Manager} \\
\textbf{Name:} Markus Müller \\
\textbf{Age:} 45 \\
\textbf{Role:} Operations Manager \\
\textbf{Goals:} Enhance production line efficiency and safety with advanced tracking. \\
\textbf{Pain Points:} Difficulty integrating new tech with legacy systems. \\
\textbf{Buying Trigger:} Push toward Industry 4.0.

\textbf{Core Persona 2: AR/VR Product Developer} \\
\textbf{Name:} Sarah Kim \\
\textbf{Age:} 38 \\
\textbf{Role:} Lead Developer \\
\textbf{Goals:} Improve 3D tracking for seamless VR interaction. \\
\textbf{Pain Points:} Limited by current tracking precision. \\
\textbf{Buying Trigger:} Preparing for a major product launch.

\textbf{Peripheral Persona 3: Security Systems Integrator} \\
\textbf{Name:} Peter Schmidt \\
\textbf{Age:} 50 \\
\textbf{Role:} Senior Security Consultant \\
\textbf{Goals:} Offer advanced security with real-time tracking. \\
\textbf{Pain Points:} Balancing technology with budget and regulations. \\
\textbf{Buying Trigger:} Increased demand for enhanced security.


\chapter{Solution Idea}

\section{Hardware}

\subsection{Computer}
The core idea is to perform image processing locally on each unit, thereby eliminating the need to transmit large volumes of raw image data to a central processing unit. This decentralized approach reduces the complexity of high-bandwidth data transfers and ensures that only the essential results, such as computational outputs, are transmitted. By evaluating single-board computers, the goal is to identify a cost-effective option that provides sufficient computational power for these local tasks. This approach not only streamlines data flow but also enhances scalability and independence between the stations.
% Responsible: Lukas Krahbichler

\subsection{Camera}
The selected camera must be compatible with the chosen single-board computer and provide high resolution to enable accurate tracking over greater distances. A 4K camera is proposed, as higher resolution theoretically extends the effective range of tracking. This choice balances precision and affordability, ensuring the system's effectiveness without unnecessary costs.
% Responsible: Lukas Krahbichler

\subsection{Display}
The primary station will include a display for visualizing tracked drone data. The visualization is one of the system's primary goals and will be developed as part of the programming section. The parameters for the display, such as resolution (Full HD) and size (8 to 12 inches), were secondary considerations compared to compatibility and affordability. To reduce costs, the display will only be included in the primary station, ensuring that it provides sufficient functionality for monitoring without adding unnecessary expenses.
% Responsible: Lukas Krahbichler

\subsection{Power Supply}
The proposed solution involves using an off-the-shelf power bank system to supply energy to all components, including the single-board computer, camera, display, and calibration hardware. This approach avoids the complexity of designing and building a custom battery management system, saving development time and effort. The power bank should have adequate output to power all components reliably and sufficient capacity to operate the system for a reasonable duration, although extended battery life is not a primary focus.
% Responsible: Lukas Krahbichler

\subsection{Data Transfer}
The idea is to implement local radio communication as the primary data transfer medium between the stations. This ensures independence from external networks, such as cellular systems, enhancing both security and operational reliability. By avoiding reliance on external infrastructure, the system becomes more robust and adaptable to various operational scenarios.
% Responsible: Lukas Krahbichler

\subsection{Calibration}
Calibration determines the relative positions and orientations of the ground stations, essential for accurate 3D drone tracking. Unlike competitors who use GNSS with RTK, this system aims to achieve similar precision through a more cost-effective and fully local approach. 

The calibration hardware, integrated onto a custom PCB, could include:
\begin{itemize}
	\item Power Delivery
	\item Time-of-Flight (ToF) Laser
	\item Communication modules
	\item Stepper motors
	\item Servo motors
	\item Gyroscope/Magnetometer/Accelerometer (9DOF)
	\item End switches
	\item Microcontroller
\end{itemize}

During calibration, approximate directions could be determined using the communication system, supplemented by precise distance measurements from the ToF laser. These measurements define the relative positions and angles of the stations, forming the foundation for accurate drone tracking.


\section{Housing}

The initial proposition of the housing focused on following principles:
\begin{itemize}
	\item Sturdiness
	\item Size
	\item Airflow (Cooling) for the Computer
\end{itemize}

% Responsible: Prantl Niclas

\section{Programming}

\subsection{Calibration}
% Responsible: Lukas Krahbichler

\subsection{Camera Tracking}
Being the key element in this project, the cameras should be able to detect and track drones mid air and calculate their relative angle to the ground station. By already knowing the angle the camera is facing, this can be done by reversing the fish-eye effect of the camera and then multiplying the relative x and y position in the image by the cameras FOV.

% Responsible: Lukas Krahbichler

\subsection{Data Transfer}

% Responsible: Lukas Krahbichler

\subsection{3D Angle Calculations}
Having already calculated the relative angles for each ground station, the individual angles are being combined by using simple trigonometry. 

% Responsible: Prantl Niclas


\subsection{3D Visualization}
Operating independently from the tracking suite, the visualization system retrieves data through network sockets. It provides a comprehensive real-time display of all three ground stations and their respective cameras. When a target is detected, the system dynamically renders lines extending from each station to the target, visually representing the tracking process. Additionally, a sphere is displayed at the calculated target position, with its size indicating the accuracy of the estimation, ensuring clear situational awareness.
% Responsible: Prantl Niclas
 

\chapter{Solution}

\section{Hardware}

\subsection{Computer}

\begin{itemize}
	\item NVIDIA Jetson Nano: too expensive and focuses on AI-Power
	\item ASUS Tinker Board S: not enough processing power?
	\item ArmSom Sige7: Good value, good processing performance
\end{itemize}


\subsection{Camera}

When choosing the ArmSom Sige7, we reviewed their other hardware to guarantee compatibility, rather than relying on third-party components. We found that they provide a 4K camera module that is fully compatible and reasonably priced.

\subsection{Display}

We found that ArmSom offers a 10.1-inch Full HD display that is fully compatible and reasonably priced.

\subsection{Power Supply}

Not sure yet (power bank didn't work)

\subsection{Data Transfer}

Use radio communication for interdependence and security. 

\subsection{Calibration}

For calibration the main station can rotate its head (camera, laser, communication) in 2 axes (pitch and yaw).
Roll is not possible (idea was to compensate with motorized feets). We have decided that we will just correct the inclination of the station in the calculation. (measured with gyro/compass)
For distance measurement: Precise ToF Laser with enough range (30 meter at least)
We discarded the use of radio direction finding for approximate direction and have decided to use the camera instead, which should be possible since we already need line of sight for the ToF Laser. (Additionally the secondary stations are full orange with should be easier to spot in a green field).

Calibration/start-up procedure:

- Calibrate yaw stepper (end-switches)
- Calibrate pitch steppers (end-switches)
- Move motors to neutral position
- Connect with other stations (all stations in neutral position)
- Get 3d magnetic orientation of every station
- Find estimate directions by sweeping yaw on primary station and looking for other stations with camera (secondaries will perhaps help in this process?)
- Measuare exact distance and direction (2 axes) with ToF Laser (direction can be read by steps from end-switch and gyro/compass)
- Move all cameras to perfect position for best maximum coverage in air
- Calculate exact relative positions/angles of cameras.

\section{Housing}

\subsection{Primary Station Housing}

\subsection{Secondary Station Housing}

\section{Programming}

\subsection{3D Angle Calculations}

\subsection{Camera Tracking}

\subsection{Data Transfer}

\subsection{Calibration}

\subsection{3D Visualization}

 
\chapter{Conclusion}

This diploma thesis focused on the development of a ground-based 3D drone tracking system using three camera-equipped stations. The primary goal was to track drones without the need for expensive onboard tracking hardware, making drone applications more accessible and cost-effective for agricultural use. The system was designed to determine drone positions by processing images from three ground stations and displaying tracking data through a 3D visualization interface.

Compared to existing solutions, our approach eliminates the need for costly GNSS/RTK systems or onboard tracking modules, significantly reducing operational costs and hardware complexity. The use of ground-based tracking enhances scalability, allowing multiple drones to be monitored without additional modifications. Furthermore, independence from external network infrastructure, such as 5G, ensures reliability in remote agricultural areas.

The significance of this work lies in its potential to lower the cost barrier for implementing drone tracking technology, making precision farming more accessible to small and medium-sized agricultural enterprises. Additionally, it explores methods for accurate multi-station synchronization and 3D position calculation.

Key findings from this work include the successful implementation of a custom approximation algorithm for determining 3D positions based on known station locations and relative angles. The modular software architecture ensures adaptability for further enhancements, such as refining calibration procedures or integrating alternative tracking methods.

Despite these achievements, challenges limited the completion of a fully functional prototype. Hardware issues, including an unreliable single-board computer and power supply inconsistencies, hindered full system integration and testing. Additionally, the current design is not waterproof, which presents limitations for outdoor agricultural use. Future improvements should focus on selecting a more reliable computing platform, and enhancing system durability.

The developed system lays a foundation for further research and industrial applications. With refinement, it could be deployed in areas such as drone-based surveillance, automated infrastructure monitoring, or environmental analysis. The methodology and findings contribute to ongoing advancements in ground-based tracking technologies, paving the way for cost-effective and scalable solutions.


\chapter*{Acknowledgments}

We would like to express our deepest gratitude to our supervising teachers, MMag.{}^{a} \textnormal{ Egger Eva-Maria, MA}, Dipl.-Ing. Götsch Leopold, PMM and Mag. \ phil.\ Jank Andreas}, for their invaluable guidance, insightful feedback, and unwavering support throughout the development of this thesis.

We are also profoundly thankful to our families, especially our parents, for their continuous support, both emotionally and financially, which has made it possible for us to pursue our education at this institution. Living away from home during the week in a  student dormitory, due to the significant distance from our school, has presented its own set of challenges.

Lastly, we extend our appreciation to our classmates, especially Reiter Matteo, and friends who have been a source of inspiration and assistance throughout this project.



% \input{content/latex_beispiele} % 3D-Object-Tracking-Documentationfach auskommentieren für die tatsächliche Arbeit

\cleardoublepage
\printnoidxglossary[type=acronym]
\printacronyms
\addcontentsline{toc}{chapter}{Acronyms}

\cleardoublepage
\listoftables
\addcontentsline{toc}{chapter}{List of Tables}

\cleardoublepage
\listoffigures
\addcontentsline{toc}{chapter}{List of Figures}

\cleardoublepage
\lstlistoflistings
\addcontentsline{toc}{chapter}{Listings}

\cleardoublepage
\appendix                       %% closes main document, appendix follows until end; only available in book-classes
\chapter*{Appendix}

\begin{table}[H]
	\centering
	\renewcommand{\arraystretch}{1.4} % Increase row spacing
	\setlength{\tabcolsep}{8pt}      % Adjust column spacing
	\begin{tabular}{p{0.3\textwidth} p{0.65\textwidth}}
		\hline
		\textbf{Attachment} & \textbf{Description} \\
		\hline
		Hardware-Drivers (C++)
		& Hardware interface drivers for each station, enabling control with sensors, motors, and calibration hardware.\\
		
		Arduino Nano Firmware (C++)
		& Code running on the Arduino Nano to manage calibration routines, stepper/servo controls, and basic communication.\\
		
		Tracking-Software (Python
		& Executed on the primary station to gather and combine data from all stations for drone position calculation.\\
		
		GUI (Python, Ursina)
		& Three-dimensional visualization interface that renders the ground stations and tracked drone positions in real time.\\
		
		Python-Tools
		& Common libraries and modules containing shared data types, vector operations, and support functions for both tracking and visualization.\\
		
		PCB (Altium/Gerber)
		& Complete Altium Designer project files and Gerber exports for manufacturing the custom PCB.\\
		
		Housing STLs
		& Three-dimensional model files (\texttt{.stl}) for printing the station enclosures and adjustable camera mounts.\\
		
		Mind Map (PDF/Image)
		& Initial brainstorming document outlining project ideas and scope.\\
		\hline
	\end{tabular}
	\caption{List of Appendices and Files}
	\label{tab:appendixfiles}
\end{table}

 %\addpart*{Appendix}

\addcontentsline{toc}{chapter}{Appendix}


\nocite{*} %Es werden auf nicht referenzierte Literaturstellen aufgelistet
\bibliography{references}

\end{document}
