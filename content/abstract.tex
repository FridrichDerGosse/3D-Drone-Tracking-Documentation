\chapter*{Abstract / Kurzfassung}
\label{cha:abstract}

This diploma thesis explores the development of a ground-based 3D drone tracking system that does not rely on expensive hardware installed on each drone. Three camera-equipped ground stations capture images of drones in flight, calculate their positions in three-dimensional space, and present the results in a user-friendly graphical interface. The project’s main goal is to provide a more cost-effective alternative to traditional drone tracking systems, making the technology more accessible for applications such as agriculture, where high costs can be a limiting factor.

Central to this approach are a calibration routine, a approximation algorithm for calculating positions, and local data processing. Early experiments confirmed that relative station angles and coordinates can, in principle, deliver valid tracking data. However, hardware issues — particularly with the chosen mini-computer — prevented the completion of a fully integrated prototype.

Despite these challenges, the concept demonstrates clear potential for drone tracking without the need for additional modules or external networks. Future improvements could focus on enhancing hardware reliability, developing a weatherproof housing, and further refining accuracy. Overall, this system establishes a foundation for a cost-effective alternative to conventional drone tracking methods and expands the possibilities for broader adoption in cost-sensitive applications.

\vspace{1cm}

Diese Diplomarbeit beschäftigt sich mit der Entwicklung eines bodengebundenen 3D-Drohnen-Tracking-Systems, das ohne teure, in der Drohne selbst verbaute Hardware auskommt. Drei mit Kameras ausgestattete Bodenstationen erfassen Drohnen im Flug, berechnen ihre Position im dreidimensionalen Raum und stellen die Ergebnisse in einer benutzerfreundlichen grafischen Oberfläche dar. Ziel des Projekts ist es, eine kosteneffiziente Alternative zu herkömmlichen Drohnen-Ortungssystemen bereitzustellen und damit den Einsatz dieser Technologie insbesondere in Bereichen wie der Landwirtschaft zu ermöglichen, wo hohe Kosten oft eine Hürde darstellen.

Das Konzept basiert auf einer präzisen Kalibrierung der Stationen, einem Algorithmus zur Positionsberechnung sowie einer lokalen Datenverarbeitung direkt in den Bodenstationen. Erste Tests zeigten, dass sich durch die Winkel- und Distanzmessung zwischen den Stationen eine zuverlässige Grundlage für die Ortung schaffen lässt. Allerdings verhinderten technische Probleme – insbesondere mit dem verwendeten Mini-Computer – die Umsetzung eines vollständig integrierten Prototyps.

Trotz dieser Herausforderungen zeigt das entwickelte Konzept deutlich, dass eine zuverlässige Drohnenortung ohne zusätzliche Module oder externe Netzwerke grundsätzlich machbar ist. Künftige Weiterentwicklungen könnten sich auf eine robustere Hardware, ein wetterfestes Gehäuse und eine höhere Genauigkeit konzentrieren. Insgesamt bildet diese Arbeit die Grundlage für eine kostengünstige Alternative zu bestehenden Tracking-Systemen und eröffnet neue Möglichkeiten für eine breitere Nutzung der Drohnentechnologie.
