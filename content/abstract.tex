\chapter*{Abstract / Kurzfassung}
\label{cha:abstract}

This diploma thesis explores the development of a ground-based 3D drone tracking system that does not rely on expensive hardware installed inside each drone. Three camera-equipped ground stations capture images of drones in flight, calculate their positions in three-dimensional space, and present the results in a user-friendly interface. The project’s main goal is to provide a more affordable alternative for smaller-scale applications—such as agricultural use—where traditional drone technology may be cost-prohibitive.

Central to this approach are synchronized calibration routines, a approximation algorithm for calculating positions, and locally managed data processing. Early experiments confirmed that relative angles and known station coordinates can, in principle, deliver valid tracking data. However, hardware issues—particularly with the chosen mini-computer—prevented the completion of a fully integrated prototype.

Despite these setbacks, the concept shows clear potential for drone tracking without requiring additional modules or external networks. Future work could address hardware reliability and weatherproof housing. Overall, this system lays the groundwork for a cost-effective alternative to conventional drone tracking methods and opens up new possibilities for broader adoption in budget-sensitive areas.

\vspace{1cm}

Diese Diplomarbeit beschäftigt sich mit der Entwicklung eines bodengestützten 3D-Drohnen-Ortungssystems, das keine kostspieligen Bauteile in der Drohne selbst benötigt. Drei Kamerastationen am Boden erfassen die Fluggeräte, berechnen deren räumliche Position und visualisieren das Ergebnis in einer benutzerfreundlichen Oberfläche. Hauptziel ist es, insbesondere in kleineren Anwendungsbereichen—wie etwa der Landwirtschaft—eine erschwingliche Alternative zu schaffen, bei der herkömmliche Drohnen-Technik oft zu teuer wäre.

Das Konzept basiert auf synchronisierten Kalibrierverfahren, einem Algorithmus zur Positionsbestimmung und lokaler Datenverarbeitung in den Bodenstationen. Erste Versuche zeigten, dass Winkelmessungen und bekannte relative Stationskoordinaten prinzipiell verlässliche Ortungsdaten liefern können. Allerdings führten Hardwareprobleme—insbesondere mit dem eingesetzten Mini-Computer—dazu, dass kein vollständig integrierter Prototyp realisiert werden konnte.

Trotz dieser Schwierigkeiten unterstreicht unsere Arbeit, dass eine Drohnenortung ohne zusätzliche Module oder externe Netzwerke grundsätzlich möglich ist. Zukünftige Arbeiten könnten sich auf zuverlässigere Hardware und ein wetterfestes Gehäuse konzentrieren. Insgesamt legt diese Arbeit den Grundstein für eine preisgünstige Alternative zu gängigen Ortungs-Systemen und eröffnet neuen Nutzergruppen den Zugang zur Drohnentechnologie.
