\chapter{Conclusion}

This diploma thesis focused on the development of a ground-based 3D drone tracking system using three camera-equipped stations. The primary goal was to track drones without the need for expensive onboard tracking hardware, making drone applications more accessible and cost-effective for agricultural use. The system was designed to determine drone positions by processing images from three ground stations and displaying tracking data through a 3D visualization interface.

Compared to existing solutions, our approach eliminates the need for costly GNSS/RTK systems or onboard tracking modules, significantly reducing operational costs and hardware complexity. The use of ground-based tracking enhances scalability, allowing multiple drones to be monitored without additional modifications. Furthermore, independence from external network infrastructure, such as 5G, ensures reliability in remote agricultural areas.

The significance of this work lies in its potential to lower the cost barrier for implementing drone tracking technology, making precision farming more accessible to small and medium-sized agricultural enterprises. Additionally, it explores methods for accurate multi-station synchronization and 3D position calculation.

Key findings from this work include the successful implementation of a custom approximation algorithm for determining 3D positions based on known station locations and relative angles. The modular software architecture ensures adaptability for further enhancements, such as refining calibration procedures or integrating alternative tracking methods.

Despite these achievements, challenges limited the completion of a fully functional prototype. Hardware issues, including an unreliable single-board computer and power supply inconsistencies, hindered full system integration and testing. Additionally, the current design is not waterproof, which presents limitations for outdoor agricultural use. Future improvements should focus on selecting a more reliable computing platform, and enhancing system durability.

The developed system lays a foundation for further research and industrial applications. With refinement, it could be deployed in areas such as drone-based surveillance, automated infrastructure monitoring, or environmental analysis. The methodology and findings contribute to ongoing advancements in ground-based tracking technologies, paving the way for cost-effective and scalable solutions.
