\chapter{Introduction}

This diploma thesis focuses on developing a ground-based 3D drone-tracking system that avoids the need for specialized onboard hardware. By combining three calibrated stations with local image processing, the system aims to provide a cost-effective alternative for operators who prefer not to modify standard drones. The core research question is whether such a fully ground-based setup can deliver reliable positional data without raising the complexity or expense of each aircraft.

Our motivation arises from an interest in aerial systems and tracking technologies, particularly in settings where budget constraints limit the viability of advanced equipment. By integrating custom hardware design, minimal calibration procedures, and flexible software solutions, we intend to create a practical infrastructure for various applications.

The thesis is organized as follows: “Detailed Task Description” clarifies specific objectives, responsibilities, and deliverables, defining the scope of work. “State of the Art: Market Analysis” then reviews existing commercial and research-based drone-tracking methods, underlining opportunities for cost savings and technical advantages. Next, the “Solution Idea” describes planned hardware, calibration, and networking strategies. The “Solution” chapter discusses practical implementation steps, including prototype testing and preliminary findings. Finally, the "Conclusion" discusses remaining limitations, and proposes directions for future enhancement.

\section{Detailed Task Description}

Main goal: Track drones with multiple ground stations

\subsection{Hardware}

\subsubsection{Computer}

\textbf{Responsible:} Krahbichler Lukas

Select hardware capable of efficiently handling the required image processing and running the \acrfull{3d}-Visualization.

\subsubsection{Camera} % This heading is not Tracking because we chose to use a camera for tracking in advance

\textbf{Responsible:} Krahbichler Lukas

Select, procure, and set up a suitable camera.

\subsubsection{Display} % The goal of this is to make the system interactable and user-friendly. As we chose to program a 3D-Visu with Ursina in Python in advance the heading is "Display" rather than "User-Interface" or something similar

\textbf{Responsible:} Krahbichler Lukas

Select and integrate a display for visualization, ensuring compatibility with other hardware components.

\subsubsection{Power Supply}

\textbf{Responsible:} Krahbichler Lukas

Design or select a power supply system that meets the requirements of all hardware components to ensure stable and efficient operation.

\subsubsection{Data Transfer}

\textbf{Responsible:} Krahbichler Lukas

Select and test a secure and fast communication medium for data transfer between the stations.

\subsubsection{Calibration}

\textbf{Responsible:} Krahbichler Lukas

Select and integrate calibration hardware essential for precise positioning and synchronization of the stations.

\subsection{Housing}

\textbf{Responsible:} Prantl Niclas

Design, test, and build housing for the primary station and secondary stations, incorporating all components.

\subsection{Programming}

\subsubsection{Development Environment}

\textbf{Responsible:} Prantl Niclas

Ensure a consistent and efficient development setup.

\subsubsection{Hardware Drivers}

\textbf{Responsible:} Prantl Niclas

Develop a modular and reliable driver system.

\subsubsection{Calibration}

\textbf{Responsible:} Krahbichler Lukas

Create software to perform calibration procedures, accurately calculating relative positions and rotations of the stations.

\subsubsection{Camera Tracking}

\textbf{Responsible:} Krahbichler Lukas

Implement software to track drones within the camera's output stream.

\subsubsection{Data Transfer}

\textbf{Responsible:} Krahbichler Lukas

Develop and implement a system to synchronize data transfer from secondary stations to the main station.

\subsubsection{3D Angle Calculations}

\textbf{Responsible:} Prantl Niclas

Develop algorithms to calculate drone positions based on data from the stations.

\subsubsection{3D Visualization}

\textbf{Responsible:} Prantl Niclas

Program a \acrshort{3d} visualization interface to display tracked drones, integrating data from all stations.
