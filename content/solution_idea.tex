\chapter{Solution Idea}

\section{Hardware}

\subsection{Computer}
The core idea is to perform image processing locally on each unit, thereby eliminating the need to transmit large volumes of raw image data to a central processing unit. This decentralized approach reduces the complexity of high-bandwidth data transfers and ensures that only the essential results, such as computational outputs, are transmitted. By evaluating single-board computers, the goal is to identify a cost-effective option that provides sufficient computational power for these local tasks. This approach not only streamlines data flow but also enhances scalability and independence between the stations.
% Responsible: Lukas Krahbichler

\subsection{Camera}
The selected camera must be compatible with the chosen single-board computer and provide high resolution to enable accurate tracking over greater distances. A 4K camera is proposed, as higher resolution theoretically extends the effective range of tracking. This choice balances precision and affordability, ensuring the system's effectiveness without unnecessary costs.
% Responsible: Lukas Krahbichler

\subsection{Display}
The primary station will include a display for visualizing tracked drone data. The visualization is one of the system's primary goals and will be developed as part of the programming section. The parameters for the display, such as resolution (Full HD) and size (8 to 12 inches), were secondary considerations compared to compatibility and affordability. To reduce costs, the display will only be included in the primary station, ensuring that it provides sufficient functionality for monitoring without adding unnecessary expenses.
% Responsible: Lukas Krahbichler

\subsection{Power Supply}
The proposed solution involves using an off-the-shelf power bank system to supply energy to all components, including the single-board computer, camera, display, and calibration hardware. This approach avoids the complexity of designing and building a custom battery management system, saving development time and effort. The power bank should have adequate output to power all components reliably and sufficient capacity to operate the system for a reasonable duration, although extended battery life is not a primary focus.
% Responsible: Lukas Krahbichler

\subsection{Data Transfer}
The idea is to implement local radio communication as the primary data transfer medium between the stations. This ensures independence from external networks, such as cellular systems, enhancing both security and operational reliability. By avoiding reliance on external infrastructure, the system becomes more robust and adaptable to various operational scenarios.
% Responsible: Lukas Krahbichler

\subsection{Calibration}
Calibration is the process of determining the relative positions and orientations of the ground stations, which is essential for accurately calculating the drone's position during operation. Unlike competitors who utilize GNSS with RTK for positioning, this system aims to achieve similar results through a simpler, more cost-effective, and entirely local solution. The calibration hardware, integrated onto a custom PCB, could include components such as:
\begin{itemize}
	\item Power Delivery
	\item Time-of-Flight (ToF) Laser
	\item Communication modules
	\item Stepper motor
	\item Servo motor
	\item Gyroscope/Magnetometer/Accelerometer (9DOF)
	\item End switches
	\item Micro-Controller
\end{itemize}
The ToF laser, which is only present in the primary station, is a key component for precise distance measurement. During calibration, approximate directions could be determined using the communication system, supplemented by detailed measurements with the ToF laser. These measurements establish the relative positions and angles of the stations, forming the basis for accurate drone tracking.
% Hierbei handelt es sich teilweise um die ursprüngliche idee. weitere ideen kamen erst als diese schief liefen. gehören diese trotzdem hier hin?
% Responsible: Lukas Krahbichler

\section{Housing}

% Responsible: Prantl Niclas

\section{Programming}

\subsection{3D Angle Calculations}
% Responsible: Prantl Niclas

\subsection{Camera Tracking}
% Responsible: Lukas Krahbichler

\subsection{Data Transfer}
% Responsible: Lukas Krahbichler

\subsection{Calibration}
% Responsible: Lukas Krahbichler

\subsection{3D Visualization}
% Responsible: Prantl Niclas
