\chapter{Solution Idea}

\section{Hardware}

\subsection{Computer}
The core idea is to perform image processing locally on each unit, thereby eliminating the need to transmit large volumes of raw image data to a central processing unit. This decentralized approach reduces the complexity of high-bandwidth data transfers and ensures that only the essential results, such as computational outputs, are transmitted. By evaluating single-board computers, the goal is to identify a cost-effective option that provides sufficient computational power for these local tasks. This approach not only streamlines data flow but also enhances scalability and independence between the stations.
% Responsible: Lukas Krahbichler

\subsection{Camera}
The selected camera must be compatible with the chosen single-board computer and provide high resolution to enable accurate tracking over greater distances. A 4K camera is proposed, as higher resolution theoretically extends the effective range of tracking. This choice balances precision and affordability, ensuring the system's effectiveness without unnecessary costs.
% Responsible: Lukas Krahbichler

\subsection{Display}
The primary station will include a display for visualizing tracked drone data. The visualization is one of the system's primary goals and will be developed as part of the programming section. The parameters for the display, such as resolution (Full HD) and size (8 to 12 inches), were secondary considerations compared to compatibility and affordability. To reduce costs, the display will only be included in the primary station, ensuring that it provides sufficient functionality for monitoring without adding unnecessary expenses.
% Responsible: Lukas Krahbichler

\subsection{Power Supply}
The proposed solution involves using an off-the-shelf power bank system to supply energy to all components, including the single-board computer, camera, display, and calibration hardware. This approach avoids the complexity of designing and building a custom battery management system, saving development time and effort. The power bank should have adequate output to power all components reliably and sufficient capacity to operate the system for a reasonable duration, although extended battery life is not a primary focus.
% Responsible: Lukas Krahbichler

\subsection{Data Transfer}
The idea is to implement local radio communication as the primary data transfer medium between the stations. This ensures independence from external networks, such as cellular systems, enhancing both security and operational reliability.
% Responsible: Lukas Krahbichler

\subsection{Calibration}
Calibration determines the relative positions and orientations of the ground stations, essential for accurate 3D drone tracking. Unlike competitors who use GNSS with RTK, this system aims to achieve similar precision through a more cost-effective and fully local approach. 

The calibration hardware, integrated onto a custom \acrfull{pcb}, could include:
\begin{itemize}
	\item Power Delivery
	\item Time-of-Flight (ToF) Laser
	\item Communication modules
	\item Stepper motors
	\item Servo motors
	\item Gyroscope/Magnetometer/Accelerometer (9DOF)
	\item End switches
	\item Microcontroller
\end{itemize}

During calibration, approximate directions could be determined using the communication system, supplemented by precise distance measurements from the ToF laser.


\section{Housing}

The initial proposition of the housing focused on following principles:
\begin{itemize}
	\item Sturdiness
	\item Size
	\item Airflow (Cooling) for the Computer
\end{itemize}

% Responsible: Prantl Niclas

\section{Programming}

\subsection{Development Environment}

A containerized Debian-based development environment is proposed to standardize dependencies and prevent system-specific issues.

\subsection{Hardware Drivers}

The software should be structured to efficiently manage hardware responsibilities.

\subsection{Calibration}
% Responsible: Lukas Krahbichler

\subsection{Camera Tracking}
Being the key element in this project, the cameras should be able to detect and track drones mid air and calculate their relative angle to the ground station. By already knowing the angle the camera is facing, this can be done by reversing the fish-eye effect of the camera and then multiplying the relative x and y position in the image by the cameras FOV.

% Responsible: Lukas Krahbichler

\subsection{Data Transfer}

% Responsible: Lukas Krahbichler

\subsection{3D Angle Calculations}
Having already calculated the relative angles for each ground station, the individual angles are being combined by using simple trigonometry. 

% Responsible: Prantl Niclas


\subsection{3D Visualization}
Operating independently from the tracking suite, the visualization system retrieves data through network sockets. It provides a comprehensive real-time display of all three ground stations and their respective cameras. When a target is detected, the system dynamically renders lines extending from each station to the target, visually representing the tracking process. Additionally, a sphere is displayed at the calculated target position, with its size indicating the accuracy of the estimation, ensuring clear situational awareness.
% Responsible: Prantl Niclas
